\documentclass[a4paper,10pt]{article}
\usepackage[utf8]{inputenc}
\usepackage[spanish,es-tabla]{babel}
\usepackage[top=3cm, bottom=2cm, right=1.5cm, left=3cm]{geometry}
\usepackage[usenames,dvipsnames,svgnames,table]{xcolor}

\title{Tablón Alterado}
\author{$a^nb^nc^n-MT$}
\date{}

\begin{document}
\maketitle

\section{Tabla alterada}

Aquí mostraremos la tabla alterada de manera aleatoria. Lo que se puede encontrar es que:
\begin{enumerate}
\item Se haya introducido un carácter que no pertenece al conjunto C.
\item Se haya cambiado un carácter a otro perteneciente al conjunto C.
\item Se haya cambiado la primera fila por otra aleatoria del tablón.
\item Se haya añadido un estado de más.
\item Eliminar estados para quitar el sentido del tablón, cambiándolos por otro signo.
\end{enumerate}\begin{table}[h]
\centering
\begin{tabular}{|l|l|l|l|l|l|l|l|l|l|l|l|l|}
\hline
	\#  &   $q_0$  &   a   &   b   &   c   &   B   &   B   &   B   &   B   &   B   &   B   &   d   &   \#	\\ \hline
	\#  &   X   &   $q_1$  &   b   &   c   &   B   &   B   &   B   &   B   &   B   &   B   &   B   &   \#	\\ \hline
	\#  &   X   &   X   &   $q_2$  &   c   &   B   &   B   &   B   &   B   &   B   &   B   &   B   &   \#	\\ \hline
	\#  &   X   &   $q_3$  &   X   &   X   &   B   &   B   &   B   &   B   &   B   &   B   &   B   &   \#	\\ \hline
	a   &   $q_3$  &   X   &   X   &   X   &   a   &   B   &   B   &   B   &   B   &   B   &   B   &   \#	\\ \hline
	\#  &   $q_3$  &   B   &   X   &   X   &   X   &   B   &   B   &   B   &   B   &   B   &   B   &   \#	\\ \hline
	\#  &   B   &   $q_0$  &   X   &   X   &   X   &   B   &   B   &   B   &   B   &   B   &   B   &   \#	\\ \hline
	\#  &   B   &   X   &   $q_0$  &   X   &   X   &   B   &   B   &   B   &   B   &   B   &   B   &   \#	\\ \hline
	\#  &   B   &   X   &   X   &   $q_0$  &   X   &   B   &   B   &   B   &   B   &   B   &   B   &   \#	\\ \hline
	\#  &   B   &   X   &   X   &   X   &   $q_0$  &   B   &   B   &   B   &   B   &   B   &   B   &   \#	\\ \hline
	\#  &   B   &   X   &   X   &   X   &   B   &   $q_6$  &   B   &   B   &   B   &   B   &   B   &   \#	\\ \hline
	\#  &   B   &   X   &   X   &   X   &   B   &   $q_6$  &   B   &   B   &   B   &   B   &   B   &   \#	\\ \hline
	\#  &   B   &   X   &   X   &   X   &   B   &   $q_6$  &   B   &   B   &   B   &   B   &   B   &   \#	\\ \hline
\end{tabular}
\end{table}
\section{Ventanas ilegales}
En este apartado se mostrarán las ventanas ilegales que nacen del tablón alterado.\newline\begin{table}[h!]
\centering
\begin{tabular}{|l|l|l|}
\hline
	B   &   B   &   d	\\ \hline
	B   &   B   &   B	\\ \hline
\end{tabular}
\end{table}

Se trata de la ventana cuya casilla central superior es la celda de la fila 1 y columna 11\newline
Esta ventana es ilegal porque contiene un símbolo ( d ) que no pertenece al conjunto C, el de los símbolos permitidos.
\newline
\begin{table}[h!]
\centering
\begin{tabular}{|l|l|l|}
\hline
	B   &   d   &   \#	\\ \hline
	B   &   B   &   \#	\\ \hline
\end{tabular}
\end{table}

Se trata de la ventana cuya casilla central superior es la celda de la fila 1 y columna 12\newline
Esta ventana es ilegal porque contiene un símbolo ( d ) que no pertenece al conjunto C, el de los símbolos permitidos.
\newline
\begin{table}[h!]
\centering
\begin{tabular}{|l|l|l|}
\hline
	\#  &   X   &   $q_3$	\\ \hline
	a   &   $q_3$  &   X	\\ \hline
\end{tabular}
\end{table}

Se trata de la ventana cuya casilla central superior es la celda de la fila 4 y columna 2\newline
En la primera fila, primera columna, hay un hastag y en la segunda fila no está debajo de él.\newline
\begin{table}[h!]
\centering
\begin{tabular}{|l|l|l|}
\hline
	X   &   X   &   B	\\ \hline
	X   &   X   &   a	\\ \hline
\end{tabular}
\end{table}

Se trata de la ventana cuya casilla central superior es la celda de la fila 4 y columna 5\newline
La ventana es ilegal porque, aunque aparentemente pueda parecer legal, no se ha podido llegar a ella desde ninguna transición.\newline
\begin{table}[h!]
\centering
\begin{tabular}{|l|l|l|}
\hline
	X   &   B   &   B	\\ \hline
	X   &   a   &   B	\\ \hline
\end{tabular}
\end{table}

Se trata de la ventana cuya casilla central superior es la celda de la fila 4 y columna 6\newline
La ventana es ilegal porque, aunque aparentemente pueda parecer legal, no se ha podido llegar a ella desde ninguna transición.\newline
\begin{table}[h!]
\centering
\begin{tabular}{|l|l|l|}
\hline
	B   &   B   &   B	\\ \hline
	a   &   B   &   B	\\ \hline
\end{tabular}
\end{table}

Se trata de la ventana cuya casilla central superior es la celda de la fila 4 y columna 7\newline
La ventana es ilegal porque, aunque aparentemente pueda parecer legal, no se ha podido llegar a ella desde ninguna transición.\newline
\begin{table}[h!]
\centering
\begin{tabular}{|l|l|l|}
\hline
	a   &   $q_3$  &   X	\\ \hline
	\#  &   $q_3$  &   B	\\ \hline
\end{tabular}
\end{table}

Se trata de la ventana cuya casilla central superior es la celda de la fila 5 y columna 2\newline
En la segunda fila, primera columna, hay un hastag y en la primera fila no está por encima de él.\newline
\begin{table}[h!]
\centering
\begin{tabular}{|l|l|l|}
\hline
	X   &   X   &   a	\\ \hline
	X   &   X   &   X	\\ \hline
\end{tabular}
\end{table}

Se trata de la ventana cuya casilla central superior es la celda de la fila 5 y columna 5\newline
La ventana es ilegal porque, aunque aparentemente pueda parecer legal, no se ha podido llegar a ella desde ninguna transición.\newline
\begin{table}[h!]
\centering
\begin{tabular}{|l|l|l|}
\hline
	X   &   a   &   B	\\ \hline
	X   &   X   &   B	\\ \hline
\end{tabular}
\end{table}

Se trata de la ventana cuya casilla central superior es la celda de la fila 5 y columna 6\newline
La ventana es ilegal porque, aunque aparentemente pueda parecer legal, no se ha podido llegar a ella desde ninguna transición.\newline
\begin{table}[h!]
\centering
\begin{tabular}{|l|l|l|}
\hline
	a   &   B   &   B	\\ \hline
	X   &   B   &   B	\\ \hline
\end{tabular}
\end{table}

Se trata de la ventana cuya casilla central superior es la celda de la fila 5 y columna 7\newline
La ventana es ilegal porque, aunque aparentemente pueda parecer legal, no se ha podido llegar a ella desde ninguna transición.\newline
\end{document}
